%
% Ryan Orvedahl Mar 2021

\documentclass[10pt,letterpaper]{article}

% Define new commands (shortcuts):
\input shortcuts

% allows the \hl{} command to highlight stuff
\usepackage{color,soul}

\usepackage{bm}
\usepackage{amsmath,amsfonts,amssymb}
\usepackage{cancel}
\usepackage{multirow}

\usepackage[margin=0.75in, letterpaper]{geometry}

% special colors and definitions for framing source code listings
\usepackage{listings}

\usepackage{tikz}
\newcommand*\circled[1]{\tikz[baseline=(math.base)]{
  \node[shape=circle,draw,inner sep=1pt] (math) {\ensuremath{#1}};}}

\definecolor{AntiqueWhite3}{rgb}{0.804,0.753,0.69}
\definecolor{DarkerAntiqueWhite3}{rgb}{0.659,0.635,0.612}
\definecolor{orange}{rgb}{1.0,0.65,0.0}

\lstset{%
  keywordstyle=\color{blue}\ttfamily,%
  commentstyle=\itshape\color[gray]{0.5},%
  mathescape=true,%
  basicstyle=\small\ttfamily,%
  %frameround=fttt,%
  frameround=ffff,%
  %frame=shadowbox,%
  frame=single,%
  rulesepcolor=\color{DarkerAntiqueWhite3},%
  backgroundcolor=\color{AntiqueWhite3},%
  emph={load,add_slice,save}, emphstyle=\color{blue},%
  emph={[2]In}, emphstyle=[2]\color{yellow},%
  emph={[3]Out}, emphstyle=[3]\color{orange},%
  xleftmargin=1em,
  xrightmargin=1em,
  mathescape=false}

%\usepackage{pdflscape} % landscape ability
%\usepackage{scrextend} % footnote and \footref commands

\begin{document}

%========================================================================
% create title and author
%========================================================================
\title{Various Notes/Thoughts from \shtns\ Library}
\author{Ryan Orvedahl}

\maketitle

\section{Introduction}
Real data on a spherical grid can be represented in two main ways. The first
is to specify the function value at each grid point in physical space,
made up of radius $r$, co-latitude $\theta$, and longitude $\phi$. An arbitrary
quantity would appear as
\begin{equation}
 \myvec{f}\left(\myvec{r},t\right)
\end{equation}
where $t$ is time and $\myvec{f}$ could be a vector or a scalar.
An alternate way of representing the real function data is in spectral space.
A scalar field is expanded in terms of spherical harmonics as
\begin{equation}
 f\left(\myvec{r},t\right) = \sum\limits_{\ell=0}^{\ell_\mathrm{max}}
                                 \sum\limits_{m=-\ell}^{\ell}
                C_\ell^m\left(r,t\right) Y_\ell^m\left(\theta,\phi\right)
\qquad
C_\ell^m = \int_\Omega \left(Y_\ell^m\right)^* f\left(\myvec{r},t\right)
           \,\mathrm{d}\Omega
\end{equation}
where the integration is taken over all solid angles. The spherical harmonic
transform (SHT) converts between representations in spectral space and
physical space, i.e., it converts between $f\left(\myvec{r},t\right)$
and $C_\ell^m\left(r,t\right)$. The SHT is generally a very slow process.

N.~Schaeffer has developed a SHT library that is reported to be very efficient
and fast compared to other third party SHT libraries. This library is called
\shtns\footnote{https://nschaeff.bitbucket.io/shtns} and it's properties are
described below.

\section{Definitions}
The spherical harmonic ``synthesis'' is the evaluation of the sum
\begin{equation}
 f\left(\myvec{r},t\right) = \sum\limits_{\ell=0}^{\ell_\mathrm{max}}
                                 \sum\limits_{m=-\ell}^{\ell}
                C_\ell^m\left(r,t\right) Y_\ell^m\left(\theta,\phi\right)
\end{equation}
which transforms from spectral space to physical space. This is accomplished by
swapping the order of the sums and expanding the spherical harmonics
\begin{equation}
 f\left(\myvec{r},t\right) =
           \sum\limits_{m=-\ell_{\mathrm{max}}}^{\ell_\mathrm{max}}
                 \sum\limits_{\ell=|m|}^{\ell_\mathrm{max}}
                 C_\ell^m Y_\ell^m
          =
           \sum\limits_{m=-\ell_{\mathrm{max}}}^{\ell_\mathrm{max}}
            \left(
                 \sum\limits_{\ell=|m|}^{\ell_\mathrm{max}}
                    C_\ell^m A_\ell^m P_\ell^m\left(\cos\theta\right)
            \right)
              e^{i m\phi}
\end{equation}
where $A_\ell^m$ is the normalization that makes the spherical harmonics orthonormal
and $P_\ell^m$ are the associated Legendre functions. The above is just a
regular Fourier transform of the stuff inside the parentheses.
The only remaining task in the ``synthesis'' step is to evaluate
\begin{equation}
 f_m\left(r,\theta,t\right) =
                 \sum\limits_{\ell=|m|}^{\ell_\mathrm{max}}
                    C_\ell^m A_\ell^m P_\ell^m\left(\cos\theta\right)
\end{equation}
on the given collocation points in co-latitude.

The ``analysis'' step in the SHT is the evaluation of the integral
\begin{equation}
  C_\ell^m\left(r,t\right) =
                  \int_\Omega \left(Y_\ell^m\right)^* f\left(\myvec{r},t\right)
           \,\mathrm{d}\Omega
\end{equation}
which transforms from physical space to spectral space.
The integral in $\phi$ is a simple Fourier transform
\begin{equation}
  C_m\left(r,\theta,t\right) =
                  \int_0^{2\pi} f\left(\myvec{r},t\right)e^{-i m \phi}
                  \,\mathrm{d}\phi
\end{equation}
and the remaining piece is
\begin{equation}
  C_\ell^m\left(r,t\right) =
                  \int_0^{\pi}
              C_m\left(r,\theta,t\right)A_\ell^mP_\ell^m\left(\cos\theta\right)
                  \,\sin\theta\,\mathrm{d}\theta
\end{equation}
This becomes a simple sum using the appropriate quadrature rule on the collocation points
in co-latitude
\begin{equation}
  C_\ell^m\left(r,t\right) =
           \sum\limits_{k=1}^{N_\theta}
              C_m\left(r,\theta_k,t\right)A_\ell^mP_\ell^m\left(\cos\theta_k\right)
              w_k
\end{equation}
where $w_k$ are the weights evaluated at the $k$-th collocation point.

\section{Optimizations}
\begin{itemize}
  \item Uniform grid in longitude on the interval $[0,2\pi)$
  \item Legendre-based grid in latitude on the interval $(0,\pi)$
  \item FFTW for the Fourier transform
  \item Complex-conjugate symmetry for real data
  \item Mirror symmetry: spherical harmonics display symmetry about the equator
  \item Precomputed $P_\ell^m\left(\cos\theta_k\right)$: save computing power
        at expense of memory bandwidth. This becomes an issue with very large
        $\ell_\mathrm{max}$ simulations.
  \item On-the-fly: evaluate $P_\ell^m\left(\cos\theta_k\right)$ recursively.
        Greatly reduce memory usage and is required
        on very large $\ell_\mathrm{max}$ simulations.
        The recursion coefficients only depend on $\ell$ and $m$ and are
        precomputed/stored. Vector instructions on modern computers can make this
        approach faster than the precomputed $P_\ell^m$ method, even for moderate
        $\ell_\mathrm{max}$.
  \item Multithreading: make use of OpenMP and multi-threaded FFTW
\end{itemize}
The last optimization includes behavior near the poles. Spherical harmonics with
large $m$ decrease exponentially near the poles. So one of the integrals in the
``analysis'' step can be reduced to
\begin{equation}
  C_\ell^m\left(r,t\right) =
                  \int_{\theta_0^{\ell m}}^{\pi-\theta_0^{\ell m}}
              C_m\left(r,\theta,t\right)A_\ell^mP_\ell^m\left(\cos\theta\right)
                  \,\sin\theta\,\mathrm{d}\theta
\end{equation}
where $\theta_0^{\ell m} \geq 0$ is the threshold below which $P_\ell^m$ is considered
zero. A similar optimization can be made during one of the ``synthesis'' integrals.
For the best results a threshold that does not depend on $\ell$ should be used. This
can lead to a 5\%-20\% speed increase depending on desired accuracy and $\ell_\mathrm{max}$.

\section{Installation}
The install process is a common ``configure-make-make install'' prescription. The
magic happens in what choices are made during the configure command.
\subsection{Configure: Installation Location}
 \begin{itemize}
  \item ``-{}-prefix=PREFIX'' where to install the architecture-independent
        files, defaults to /usr/local
  \item ``-{}-exec-prefix=EPREFIX'' where to install the architecture-dependent
        files, defaults to PREFIX
  \item ``-{}-libdir=DIR'' where object code libraries go, defaults to EPREFIX/lib
  \item ``-{}-includedir=DIR'' where C header files will be, defaults to PREFIX/include
 \end{itemize}
\subsection{Configure: Optimizations}
 \begin{itemize}
  \item ``-{}-enable-openmp'' enables multi-threading with OpenMP
  \item ``-{}-enable-knl'' enables compilation for Xeon Phi (KNL)
  \item ``-{}-enable-cuda[=kepler$|$pascal$|$volta$|$ampere]'' enables compilation for
        Nvidia GPU, with optional specification of the architecture. This requires
        a CUDA library, cudart(?) I think
  \item ``-{}-enable-python'' builds the Python interface
  \item ``-{}-disable-simd'' will turn off vector extensions (AVX, AVX2, etc.)
  \item ``-{}-enable-mkl'' will call FFTW through MKL, giving a slight performance boost,
        BUT it is not thread-safe and the user is responsible for initializing/calling
        \shtns\ correctly
 \end{itemize}
\subsection{Configure: Environment Variables}
 \begin{itemize}
  \item PYTHON - the python interpreter, defaults to `python'
  \item CC - C compiler command
  \item CFLAGS - C compiler flags
  \item LDFLAGS - linker flags, e.g., -L$<$lib dir$>$ for a nonstandard library
  \item LIBS - libraries to pass to the linker
  \item CPP - C preprocessor
  \item CPPFLAGS - C/C++ preprocessor flags, e.g., -I$<$include dir$>$ for headers in
        a nonstandard location
  \item FC - Fortran compiler command
  \item FCFLAGS - Fortran compiler flags
 \end{itemize}
\subsection{Makefile}
The configure command will run a bunch of checks and write the appropriate Makefile.
If the PREFIX was not specified, you can change it in the Makefile now. If a C compiler
other than gcc was used, you may want to check the Makefile for consistency (this note
is in the \shtns\ docs, but I use gcc).

When inspecting the Makefile ``install'' instruction, all it was going to do was run
`python setup.py install'. This must be run as root, to avoid this simply run
`python setup.py install -{}-user'. Other combinations of configure commands may
produce a different install operation.

\section{Initialization}
\shtns\ is fairly general and can handle quite a few different truncated transforms.
There are five main controls to set (\rayleigh\ can get away with one maybe two):
\begin{itemize}
 \item ``lmax'' - the maximum spherical harmonic degree
 \item ``mres'' - the azimuthal periodicity given by $2\pi/$mres
 \item ``mmax'' - the maximum azimuthal wave number will be mmax*mres
 \item ``nlat'' - the number of points in latitude
 \item ``nphi'' - the number of points in latitude
\end{itemize}
In \rayleigh, we choose only lmax (and a dealiasing factor). Once that is chosen, we can
set
\begin{itemize}
 \item mmax = lmax
 \item mres = 1
 \item nlat = $<$dealias factor$>$ $\times$ (lmax + 1)
 \item nphi = 2*nlat
\end{itemize}
Next we choose the normalization ``sht\_orthonormal'', although others are possible.
The grid choice comes next, \shtns\ can do different theta grids, but we
want to choose ``sht\_gauss'' to use the faster Gauss-Legendre quadrature.
We can further choose ``sht\_phi\_contiguous'' or ``sht\_theta\_contiguous'' to
direct \shtns\ how to store internal data.

OpenMP threads need to be set before the initialization of the SHT object happens.
The SHT object is created with a call involving the lmax, mmax, mres, normalization,
and grid choices. The SHT object is a pointer/structure.
Next, the grid is built and attached to the SHT object. Here we choose the number
of latitude/longitude grid points as well as the polar optimization tolerance.
The options for the tolerance are
\begin{itemize}
 \item $\epsilon = 10^{-14}$ is VERY safe
 \item $\epsilon = 10^{-10}$ is safe
 \item $\epsilon = 10^{-6}$ is agressive, but still good accuracy
\end{itemize}
At this point the SHT has been built and initialized; it is ready to be used.

\subsection{SHT object}
A brief note about the SHT object as it stores some useful arrays and values. The
following attributes appear for the SHT object named ``SHT'' (this is not all
attributes, just the main ones):
\begin{itemize}
 \item ``SHT\%lmax'' - maximum spherical harmonic
 \item ``SHT\%mmax'' - maximum azimuthal wave number
 \item ``SHT\%mres'' - periodicity in azimuth
 \item ``SHT\%nlat'' - number of points in latitude
 \item ``SHT\%nphi'' - number of points in longitude
 \item ``SHT\%nlm'' - number of (l,m) entries/modes
 \item ``SHT\%li'' - array of size nlm, gives the degree l for the given mode index
 \item ``SHT\%mi'' - array of size nlm, gives the order m for the given mode index
 \item ``SHT\%ct'' - array of size nlat, gives cos(theta)
 \item ``SHT\%st'' - array of size nphi, gives sin(theta)
\end{itemize}
An important note is that the $\cos\theta$ grid is reversed compared to \rayleigh.

\subsection{Useful Helper Functions}
The integration weights can be found with a subroutine: ``call shtns\_gauss\_wts(SHT, w)''
where the returned array w is of size nlat.

The mode index of a (l,m) pair can be found with a function:
``lm = shtns\_lmidx(SHT, l, m)''. The reverse operation gives the degree and order for
the given mode index: ``l = shtns\_lm2l(SHT, lm)'' and ``m = shtns\_lm2m(SHT, lm)''.

\subsection{Full SHT}
The full SHT from spectral to physical space appears as a subroutine:
``call sh\_to\_spat(SHT, spec, phys)'' where spec is an array of size (nlm) and
phys is an array of size (nphi,nlat).

The full SHT from physical to spectral space appears as a subroutine:
``call spat\_to\_sh(SHT, phys, spec)'' where spec is an array of size (nlm) and
phys is an array of size (nphi,nlat).

\subsection{Legendre Transform}
A Legendre transform from physical to spectral space at fixed order m is a three
step process:
\begin{enumerate}
  \item lm\_start = shtns\_lmidx(SHT, 0, m)
  \item lm\_stop = shtns\_lmidx(SHT, lmax, m)
  \item call spat\_to\_sh\_ml(SHT, m, phys, spec(lm\_start:lm\_stop), lmax)
\end{enumerate}
Since the Legendre transform can do truncated transforms, we must specify the start/stop
of the spherical harmonic degree. Of course the azimuthal order m should be the same
for all three steps. The truncation is specified by lmax in the last argument. Here
phys is an array of size (nlat) and spec is of size (nlm).

A Legendre transform from spectral to physical space at fixed order m is a three
step process:
\begin{enumerate}
  \item lm\_start = shtns\_lmidx(SHT, 0, m)
  \item lm\_stop = shtns\_lmidx(SHT, lmax, m)
  \item call sh\_to\_spat\_ml(SHT, m, spec(lm\_start:lm\_stop), phys, lmax)
\end{enumerate}
The truncation is specified by lmax in the last argument. Here
phys is an array of size (nlat) and spec is of size (nlm).

This lends itself to a nice interface, similar to what \rayleigh\ uses where both
forward and reverse transforms both use a call to Legendre\_Transform


\subsection{Finalize \shtns}
To deallocate and free any memory objects, two calls appear:
``call shtns\_unset\_grid(SHT)'' and ``call shtns\_destroy(SHT)''.

\section{Legendre Polynomials/Transforms in \rayleigh}
The Legendre polynomials module stores all things related to $P_\ell^m$. The main objects
that it produces are the integration weights and the colocation points. The only
routines that use this module are
\begin{itemize}
  \item ProblemSize: calls the initialization routine and stores the colocation points
  \item Benchmarking: only uses the integration weights
  \item Diagnostic Interface: only uses the integration weights
  \item Legendre Transforms: $P_\ell^m(x)$, colocation points, and others
\end{itemize}
The Linear Terms Cart routine has a commented out dependency and Sphere Hybrid Space has
an explicit use-only, but the used object is never actually used anywhere else.

Here is a list of all the routines that use the Legendre Transforms module:
\begin{itemize}
  \item Parallel IO (called once)
  \item Sphere Hybrid Space (called in three spots)
  \item Diagnostics Second Derivatives (calls transform in two spots, but never uses module?)
  \item Spherical IO (explicit `use-only', but never actually used)
  \item Initial Conditions (explicit `use-only', but never actually used)
  \item Test SHT (probably really out of date?)
\end{itemize}
In all cases, only the Legendre Transform is used from the module.

\subsection{Possible Ideas Towards \shtns}
Only a very small number of routines actually use the Legendre Polynomials module,
and the ones that do really only use the integration weights and/or the grid points
(with the exception of the Legendre Transform module).

This means the \shtns\ library is probably best used as a single module that stores
both the Legendre Polynomial information as well as the Legendre Transform routines.

A preprocessor directive could be wrapped around all the use Legendre Polynomials and
the use Legendre Transform calls? Stopping there would require keeping the exact same
calling interface for the Legendre Transform, possibly done with a wrapper to \shtns.
Of we could put a preprocessor directive around all calls to the transform?

\subsection{Potential Parallel Issues}
Presumably, $\theta$ is in-processor when the Legendre transform needs to take place. If
not, that will be quite the head-stumper to figure out.
\begin{itemize}
  \item Does \rayleigh\ use the same $\ell-m$ mode ordering as \shtns?
        $\ell$-major vs. $m$-major?
  \item Are all \rayleigh\ Legendre transforms completed in $m$ space?
  \item \shtns\ does one quantity at a time $\rightarrow$ loop over $r$, $m$, and fields
\end{itemize}

\subsection{Potential Derivative Issues}
The $\cos\theta$ grid is reversed, which may cause a giant headache with the spectral
derivative routines. Or maybe not - write a test case to see how hard it is to
`un-reverse' the grid, especially when recurence relations are used. There are three
routines that do spectral derivatives, but only one is actually used by any other
files: Spectral Derivatives. Here is a list of all routines that use it:
\begin{itemize}
  \item ProblemSize: only to initialize the horizontal grid
  \item Diagnostics Second Derivatives
  \item Sphere Hybrid Space
  \item Sphere Physical Space
\end{itemize}

\subsection{Current Method}
Both the forward and reverse Legendre transforms are currently computed using the DGEMM
matrix-matrix multiply routine from BLAS. DGEMM computes
\begin{equation}
  C \leftarrow \alpha \mathrm{op}\left(A\right)\cdot\mathrm{op}\left(B\right) + \beta C
\end{equation}
with
\begin{equation}
  \mathrm{op}\left(X\right) = X
  \qquad\mathrm{or}\qquad
  \mathrm{op}\left(X\right) = X^T
\end{equation}
The left arrow indicates that the result is stored in $C$, even though $C$ can be
an input quantity. The calling sequence in Fortran is
\begin{equation}
  \mathrm{Call\ DGEMM}\left(TransA,TransB,m,n,k,\alpha,A,ldA,B,ldB,\beta,C,ldC\right)
\end{equation}
and the meaning of the various quantities is
\begin{itemize}
  \item $TransA$ : single character, either `N' or `T', to indicate if $A$ should be
        transposed before multiplication.
        `N' will produce $\mathrm{op}\left(A\right)=A$
        and `T' will produce $\mathrm{op}\left(A\right)=A^T$
  \item $TransB$ : single character, either `N' or `T', to indicate if $B$ should be
        transposed before multiplication
        `N' will produce $\mathrm{op}\left(B\right)=B$
        and `T' will produce $\mathrm{op}\left(B\right)=B^T$
  \item $m$,$n$,$k$ : describe array shapes,
        $\mathrm{op}\left(A\right)=(m,k)$,
        $\mathrm{op}\left(B\right)=(k,n)$,
        $C=(m,n)$
  \item $\alpha$ : real multiplicative scalar
  \item $A$ : storage for the array values of shape $(ldA,lA)$ where $lA=k$ when `N' and
        $lA=m$ if `T'.
  \item $ldA$ : first dimension of $A$ as declared/allocated.
  \item $B$ : storage for the array values of shape $(ldB,lB)$ where $lB=n$ when `N' and
        $lB=k$ if `T'.
  \item $ldB$ : first dimension of $B$ as declared/allocated.
  \item $\beta$ : real multiplicative scalar
  \item $C$ : storage for the array values of shape $(ldC,n)$, overwritten on exit
  \item $ldC$ : first dimension of $C$ as declared/allocated.
\end{itemize}

\subsubsection{Legendre Polynomial Storage Arrays}
The $P_\ell^m\left(\cos\theta\right)$ are stored by $\ell-m$ mode index (which is really
just one-to-one with $m$) in an array
dimensioned: p\_lm(1:n\_lm)\%data(m:l\_max, 1:n\_theta), where the lower bound of the
first axis is computed based on the mode index lm as m=m\_values(lm). The first dimension
runs from ``actual azimuthal wavenumber $m$ associated with the current mode index up to
l\_max''.

The integration weights are folded into the ip\_lm structure, which uses a similar storage
layout, but dimensions are reversed. ip\_lm(1:n\_lm)\%data(1:n\_theta, m:l\_max).
There are additional FFT weights folded in, called PTS\_normalization and STP\_normalization.

In either case, the number of actual rows/columns in the `l' dimension is l\_max - m + 1.

\subsubsection{Data Layouts Pre/Post-Transform}
The only layouts used with the Legendre transform are the `s2a/b' and `p2a/b' configurations.
The only routines that actually call the transform include
\begin{itemize}
 \item Diagnostics\_Second\_Derivatives.F90: p2b$\rightarrow$s2b \& s2a$\rightarrow$p2a
 \item Parallel\_IO.F90: only p2b$\rightarrow$s2b
 \item Sphere\_Hybrid\_Space.F90: s2a$\rightarrow$p2a \& p2b$\rightarrow$s2b
 \item Test\_SHT.F90: s2a$\leftrightarrow$p2a, p2b$\rightarrow$s2a
\end{itemize}
The main loop uses the s2a$\rightarrow$p2a followed by a transpose to p3a and a transpose
to p2b, which is followed by the p2b$\rightarrow$s2b transform. This is modeled in the
Test\_SHT.F90 routine in the Amp\_Test\_Parallel subroutine.

The s2a/b layout is similar to the p\_lm arrays; 4D arrays orderd by mode index (i.e., $m$).
The p2a/b layout is a 3D array, used mainly for the Legendre transforms.

\end{document}


