%
% Ryan Orvedahl Mar 2021

\documentclass[10pt,letterpaper]{article}

% Define new commands (shortcuts):
\input shortcuts

% allows the \hl{} command to highlight stuff
\usepackage{color,soul}

\usepackage{bm}
\usepackage{amsmath,amsfonts,amssymb}
\usepackage{cancel}
\usepackage{multirow}

\usepackage[margin=0.75in, letterpaper]{geometry}

% special colors and definitions for framing source code listings
\usepackage{listings}

\usepackage{tikz}
\newcommand*\circled[1]{\tikz[baseline=(math.base)]{
  \node[shape=circle,draw,inner sep=1pt] (math) {\ensuremath{#1}};}}

\definecolor{AntiqueWhite3}{rgb}{0.804,0.753,0.69}
\definecolor{DarkerAntiqueWhite3}{rgb}{0.659,0.635,0.612}
\definecolor{orange}{rgb}{1.0,0.65,0.0}

\lstset{%
  keywordstyle=\color{blue}\ttfamily,%
  commentstyle=\itshape\color[gray]{0.5},%
  mathescape=true,%
  basicstyle=\small\ttfamily,%
  %frameround=fttt,%
  frameround=ffff,%
  %frame=shadowbox,%
  frame=single,%
  rulesepcolor=\color{DarkerAntiqueWhite3},%
  backgroundcolor=\color{AntiqueWhite3},%
  emph={load,add_slice,save}, emphstyle=\color{blue},%
  emph={[2]In}, emphstyle=[2]\color{yellow},%
  emph={[3]Out}, emphstyle=[3]\color{orange},%
  xleftmargin=1em,
  xrightmargin=1em,
  mathescape=false}

%\usepackage{pdflscape} % landscape ability
%\usepackage{scrextend} % footnote and \footref commands

\begin{document}

%========================================================================
% create title and author
%========================================================================
\title{Various Notes from \shtns\ Library}
\author{Ryan Orvedahl}

\maketitle

\section{Introduction}
Real data on a spherical grid can be represented in two main ways. The first
is to specify the function value at each grid point in physical space,
made up of radius $r$, co-latitude $\theta$, and longitude $\phi$. An arbitrary
quantity would appear as
\begin{equation}
 \myvec{f}\left(\myvec{r},t\right)
\end{equation}
where $t$ is time and $\myvec{f}$ could be a vector or a scalar.
An alternate way of representing the real function data is in spectral space.
A scalar field is expanded in terms of spherical harmonics as
\begin{equation}
 f\left(\myvec{r},t\right) = \sum\limits_{\ell=0}^{\ell_\mathrm{max}}
                                 \sum\limits_{m=-\ell}^{\ell}
                C_\ell^m\left(r,t\right) Y_\ell^m\left(\theta,\phi\right)
\qquad
C_\ell^m = \int_\Omega \left(Y_\ell^m\right)^* f\left(\myvec{r},t\right)
           \,\mathrm{d}\Omega
\end{equation}
where the integration is taken over all solid angles. The spherical harmonic
transform (SHT) converts between representations in spectral space and
physical space, i.e., it converts between $f\left(\myvec{r},t\right)$
and $C_\ell^m\left(r,t\right)$. The SHT is generally a very slow process.

N.~Schaeffer has developed a SHT library that is reported to be very efficient
and fast compared to other third party SHT libraries. This library is called
\shtns\footnote{https://nschaeff.bitbucket.io/shtns} and it's properties are
described below.

\section{Definitions}
The spherical harmonic ``synthesis'' is the evaluation of the sum
\begin{equation}
 f\left(\myvec{r},t\right) = \sum\limits_{\ell=0}^{\ell_\mathrm{max}}
                                 \sum\limits_{m=-\ell}^{\ell}
                C_\ell^m\left(r,t\right) Y_\ell^m\left(\theta,\phi\right)
\end{equation}
which transforms from spectral space to physical space. This is accomplished by
swapping the order of the sums and expanding the spherical harmonics
\begin{equation}
 f\left(\myvec{r},t\right) =
           \sum\limits_{m=-\ell_{\mathrm{max}}}^{\ell_\mathrm{max}}
                 \sum\limits_{\ell=|m|}^{\ell_\mathrm{max}}
                 C_\ell^m Y_\ell^m
          =
           \sum\limits_{m=-\ell_{\mathrm{max}}}^{\ell_\mathrm{max}}
            \left(
                 \sum\limits_{\ell=|m|}^{\ell_\mathrm{max}}
                    C_\ell^m A_\ell^m P_\ell^m\left(\cos\theta\right)
            \right)
              e^{i m\phi}
\end{equation}
where $A_\ell^m$ is the normalization that makes the spherical harmonics orthonormal
and $P_\ell^m$ are the associated Legendre functions. The above is just a
regular Fourier transform of the stuff inside the parentheses.
The only remaining task in the ``synthesis'' step is to evaluate
\begin{equation}
 f_m\left(r,\theta,t\right) =
                 \sum\limits_{\ell=|m|}^{\ell_\mathrm{max}}
                    C_\ell^m A_\ell^m P_\ell^m\left(\cos\theta\right)
\end{equation}
on the given collocation points in co-latitude.

The ``analysis'' step in the SHT is the evaluation of the integral
\begin{equation}
  C_\ell^m\left(r,t\right) =
                  \int_\Omega \left(Y_\ell^m\right)^* f\left(\myvec{r},t\right)
           \,\mathrm{d}\Omega
\end{equation}
which transforms from physical space to spectral space.
The integral in $\phi$ is a simple Fourier transform
\begin{equation}
  C_m\left(r,\theta,t\right) =
                  \int_0^{2\pi} f\left(\myvec{r},t\right)e^{-i m \phi}
                  \,\mathrm{d}\phi
\end{equation}
and the remaining piece is
\begin{equation}
  C_\ell^m\left(r,t\right) =
                  \int_0^{\pi}
              C_m\left(r,\theta,t\right)A_\ell^mP_\ell^m\left(\cos\theta\right)
                  \,\sin\theta\,\mathrm{d}\theta
\end{equation}
This becomes a simple sum using the appropriate quadrature rule on the collocation points
in co-latitude
\begin{equation}
  C_\ell^m\left(r,t\right) =
           \sum\limits_{k=1}^{N_\theta}
              C_m\left(r,\theta_k,t\right)A_\ell^mP_\ell^m\left(\cos\theta_k\right)
              w_k
\end{equation}
where $w_k$ are the weights evaluated at the $k$-th collocation point.

\section{Optimizations}
\begin{itemize}
  \item Uniform grid in longitude on the interval $[0,2\pi)$
  \item Legendre-based grid in latitude on the interval $(0,\pi)$
  \item FFTW for the Fourier transform
  \item Complex-conjugate symmetry for real data
  \item Mirror symmetry: spherical harmonics display symmetry about the equator
  \item Precomputed $P_\ell^m\left(\cos\theta_k\right)$: save computing power
        at expense of memory bandwidth. This becomes an issue with very large
        $\ell_\mathrm{max}$ simulations.
  \item On-the-fly: evaluate $P_\ell^m\left(\cos\theta_k\right)$ recursively.
        Greatly reduce memory usage and is required
        on very large $\ell_\mathrm{max}$ simulations.
        The recursion coefficients only depend on $\ell$ and $m$ and are
        precomputed/stored. Vector instructions on modern computers can make this
        approach faster than the precomputed $P_\ell^m$ method, even for moderate
        $\ell_\mathrm{max}$.
  \item Multithreading: make use of OpenMP and multi-threaded FFTW
\end{itemize}
The last optimization includes behavior near the poles. Spherical harmonics with
large $m$ decrease exponentially near the poles. So one of the integrals in the
``analysis'' step can be reduced to
\begin{equation}
  C_\ell^m\left(r,t\right) =
                  \int_{\theta_0^{\ell m}}^{\pi-\theta_0^{\ell m}}
              C_m\left(r,\theta,t\right)A_\ell^mP_\ell^m\left(\cos\theta\right)
                  \,\sin\theta\,\mathrm{d}\theta
\end{equation}
where $\theta_0^{\ell m} \geq 0$ is the threshold below which $P_\ell^m$ is considered
zero. A similar optimization can be made during one of the ``synthesis'' integrals.
For the best results a threshold that does not depend on $\ell$ should be used. This
can lead to a 5\%-20\% speed increase depending on desired accuracy and $\ell_\mathrm{max}$.

\section{Installation}
The install process is a common ``configure-make-make install'' prescription. The
magic happens in what choices are made during the configure command.
\subsection{Configure: Installation Location}
 \begin{itemize}
  \item ``-{}-prefix=PREFIX'' where to install the architecture-independent
        files, defaults to /usr/local
  \item ``-{}-exec-prefix=EPREFIX'' where to install the architecture-dependent
        files, defaults to PREFIX
  \item ``-{}-libdir=DIR'' where object code libraries go, defaults to EPREFIX/lib
  \item ``-{}-includedir=DIR'' where C header files will be, defaults to PREFIX/include
 \end{itemize}
\subsection{Configure: Optimizations}
 \begin{itemize}
  \item ``-{}-enable-openmp'' enables multi-threading with OpenMP
  \item ``-{}-enable-knl'' enables compilation for Xeon Phi (KNL)
  \item ``-{}-enable-cuda[=kepler$|$pascal$|$volta$|$ampere]'' enables compilation for
        Nvidia GPU, with optional specification of the architecture. This requires
        a CUDA library, cudart(?) I think
  \item ``-{}-enable-python'' builds the Python interface
  \item ``-{}-disable-simd'' will turn off vector extensions (AVX, AVX2, etc.)
  \item ``-{}-enable-mkl'' will call FFTW through MKL, giving a slight performance boost,
        BUT it is not thread-safe and the user is responsible for initializing/calling
        \shtns\ correctly
 \end{itemize}
\subsection{Configure: Environment Variables}
 \begin{itemize}
  \item PYTHON - the python interpreter, defaults to `python'
  \item CC - C compiler command
  \item CFLAGS - C compiler flags
  \item LDFLAGS - linker flags, e.g., -L$<$lib dir$>$ for a nonstandard library
  \item LIBS - libraries to pass to the linker
  \item CPP - C preprocessor
  \item CPPFLAGS - C/C++ preprocessor flags, e.g., -I$<$include dir$>$ for headers in
        a nonstandard location
  \item FC - Fortran compiler command
  \item FCFLAGS - Fortran compiler flags
 \end{itemize}
\subsection{Makefile}
The configure command will run a bunch of checks and write the appropriate Makefile.
If the PREFIX was not specified, you can change it in the Makefile now. If a C compiler
other than gcc was used, you may want to check the Makefile for consistency (this note
is in the \shtns\ docs, but I use gcc).

When inspecting the Makefile ``install'' instruction, all it was going to do was run
`python setup.py install'. This must be run as root, to avoid this simply run
`python setup.py install -{}-user'. Other combinations of configure commands may
produce a different install operation.

\end{document}
